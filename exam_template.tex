\documentclass[oneside]{scrbook}
\usepackage[T2A]{fontenc}
\usepackage[utf8]{inputenc}
\usepackage[russian]{babel}
\usepackage[left=3cm, right=1.5cm, top=2cm, bottom=2cm]{geometry}
\usepackage{needspace}  % для проверки свободного места
\usepackage{array}      % для управления выравниванием в колонках
\usepackage{float}      % для указания точного местоположения таблицы

\renewcommand{\arraystretch}{1.3} % немного увеличить высоту строк таблицы для лучшей читаемости

\newcommand{\examticket}[2]{% #1 = номер билета, #2 = вопросы
  \par
  \needspace{16\baselineskip}
  \noindent
  \begin{minipage}{\textwidth}
    % === Центрированная часть: шапка и таблица ===
    {\centering
      \textbf{Негосударственное небюджетное образовательное учреждение}\\
      \textbf{«Урюпинский нетехнический техникум имени В.А. Пупкина»}

      \vspace{0.4\baselineskip}

      \small
      \begin{tabular}{|%
        >{\raggedright\arraybackslash}p{0.25\linewidth}|%
        >{\centering\arraybackslash}p{0.42\linewidth}|%
        >{\raggedleft\arraybackslash}p{0.25\linewidth}|%
      }
      \hline
      Рассмотрено ПЦК & \textbf{Экзаменационный билет №#1} & Утверждаю \\
      «Абракадабра» & по учебной дисциплине & Зам. директора \\
      протокол №354 & ОП.01 Основы ничего не делания & \rule{1cm}{0.4pt} / К.Д. Люлайка \\
      от 29.02.2144 г.& специальность 09.02.06 & 01.01.1999 г. \\
      Председатель ПЦК & «Сетевое и системное администрирование» & \\
      \rule{1cm}{0.4pt} / Бубликов А.Б. & группа(ы) АБ-10, АБ-10к & \\
      Согласовано методист & & \\
      \rule{1cm}{0.4pt} / Г.Д. Щупликов & & \\
      01.01.1999 г. & & \\
      \hline
      \end{tabular}
      \par
    } % ← закрываем группу с \centering

    % === Вопросы — выравнивание по левому краю (по умолчанию) ===
    \vspace{0.6\baselineskip}
    \normalsize
    #2 % вопросы автоматически выровнены по левому краю

    % === Подпись преподавателя — центрированная ===
    \vspace{0.5\baselineskip}
    {\centering
      Преподаватель \rule{3cm}{0.4pt}/Н.О. Труляляков\par
    }
  \end{minipage}
  \par
  \vfill
}

\begin{document}

% Билеты формируются автоматически, также билет автоматически перейдёт на новую страницу, если не влезет.

% Билет 1
\examticket{1}{
1. Предложите методологию оценки качества абстрактных идей, не имеющих никакого практического применения, количественных характеристик или даже минимального смысла, с использованием шкалы измерения, основанной на субъективных ощущениях воображаемого эксперта.

2. Объясните механизм передачи семантического содержания от несуществующего отправителя к несуществующему получателю через канал связи, который никогда не был создан и не может быть теоретически реализован в рамках известных законов физики и логики.

3. Разработайте концептуальную модель взаимодействия между воображаемыми частицами «ничтона» и «пустон» в условиях абсолютного вакуума, включая расчёт их гипотетических траекторий при отсутствии пространства и времени как таковых.
}

% Билет 2
\examticket{2}{
1. Определите граничные условия для спонтанной генерации информационных паттернов в полностью изолированной системе, лишённой каких-либо источников энергии, информации или даже самой возможности существования.

2. Обоснуйте необходимость создания единой классификации для несуществующих явлений, включая разработку таксономической шкалы с плавающими критериями идентификации, адаптируемыми под текущие потребности воображаемого наблюдателя.

3. Сравните эффективность применения метода инверсной логики к решению неразрешимых задач с альтернативным подходом, основанным на преднамеренном введении дополнительных противоречий для достижения состояния когнитивного резонанса.
}

% Билет 3
\examticket{3}{
1. Перечислите пять ключевых принципов синергетического взаимодействия между фиктивными сущностями в замкнутой системе, демонстрируя их влияние на энтропийный баланс при условии отсутствия каких-либо реальных взаимодействий.

2. Опишите алгоритм трансформации абстрактных концепций в материальные артефакты с использованием метода рекурсивной деконструкции парадоксальных утверждений и последующей их репликации в неевклидовом пространстве с отрицательной кривизной.

3. Проанализируйте корреляционную зависимость между квантовыми флуктуациями вакуума и сезонной миграцией несуществующих птиц в гиперпространственных координатах, учитывая влияние обратной связи от воображаемых наблюдателей на стабильность метрики пространства-времени.
}

\end{document}
