\documentclass[12pt]{article}
\usepackage[T2A]{fontenc}
\usepackage[utf8]{inputenc}
\usepackage[russian]{babel}
\usepackage{caption}
\usepackage{setspace}
\usepackage{titlesec}
\usepackage{enumitem}
\usepackage{geometry}
\usepackage{indentfirst} % ← делает отступ первой строки после \section
\usepackage{needspace}  % для проверки свободного места
\usepackage{pgffor}     % для цикла \foreach

\onehalfspacing
\setlist[itemize]{noitemsep, topsep=4pt, parsep=0pt, partopsep=0pt, leftmargin=1.25cm}
\setlist[enumerate]{noitemsep, topsep=4pt, parsep=0pt, partopsep=0pt, leftmargin=1.25cm}
\setitemize[1]{label=--}
\pagestyle{empty}
\titleformat{\section}{\normalfont\large\bfseries\centering}{\thesection}{1em}{}
\setlength{\parindent}{1.25cm} % стандартный отступ по ГОСТ (1.25 см)
\setlength{\parskip}{0pt}      % убрать вертикальный отступ между абзацами
\geometry{top=2cm, bottom=2cm, left=2.5cm, right=2cm} % Поля страницы

%парсер тхт файлов
\newread\myfile
\def\readline#1#2{%
  \openin\myfile=#2\relax
  \readlinehelper{#1}{1}%
  \closein\myfile
}
\def\readlinehelper#1#2{%
  \ifeof\myfile
    % Если дошли до конца — ничего не делаем
  \else
    \read\myfile to \tmp
    \ifnum#2=#1\relax
      \tmp % выводим нужную строку
    \else
      \expandafter\readlinehelper\expandafter#1\expandafter{\number\numexpr#2+1\relax}%
    \fi
  \fi
}

% Генератор билетов
\newcommand{\examticket}[2]{% #1 = номер билета, #2 = вопросы
  \par
  \needspace{16\baselineskip}
  \noindent
  \begin{minipage}{\textwidth}
  \textbf{Билет №#1}\\
    #2 % вопросы
  \end{minipage}
  \par
  \vfill
}

\makeatletter

% Накопитель
\def\competencylist{}

% Чтение CSV
\newread\compfile
\def\readcompetencies#1{%
  \openin\compfile=#1
  % Пропускаем первую строку (заголовок)
  \read\compfile to \dummyline
  % Читаем остальные
  \loop
    \read\compfile to \line
    \unless\ifeof\compfile
      \ifx\line\empty\else
        \expandafter\extractfirstfield\line\relax
      \fi
  \repeat
  \closein\compfile
}

% Извлекает всё до первой запятой
\def\extractfirstfield#1,#2\relax{%
  \ifx\competencylist\@empty
    \gdef\competencylist{#1}%
  \else
    \g@addto@macro\competencylist{, #1}%
  \fi
}

\makeatother

\makeatletter

% Вторая функция: вывод списка "comp. description"
\def\competencyitems{}

\def\readcompetencyitems#1{%
  \openin\compfile=#1
  % Пропускаем заголовок
  \read\compfile to \dummyline
  \loop
    \read\compfile to \line
    \unless\ifeof\compfile
      \ifx\line\empty\else
        \expandafter\parseitem\line\relax
      \fi
  \repeat
  \closein\compfile
}

% Парсинг строки: "OK 01,Описание" → "OK 01. Описание"
\def\parseitem#1,#2\relax{%
  \ifx\competencyitems\@empty
    \gdef\competencyitems{#1. #2}%
  \else
    \g@addto@macro\competencyitems{\par #1. #2}%
  \fi
}

\makeatother

% Удобный макрос для использования
\newcommand{\printcompetencylist}[1]{%
  \readcompetencyitems{#1}%
  \competencyitems
}

\begin{document}

% 1. Шапка — название колледжа (по центру)
\begin{center}
  {
  государственное бюджетное профессиональное образовательное учреждение\\
  «Пермский политехнический колледж имени Н.Г. Славянова»
  }
\end{center}

% 2. Блок УТВЕРЖДЕНО (слева страницы, текст внутри — по правому краю)
\begin{flushright}
    \vspace{1.5cm}
    \begin{tabular}{r@{}} % r = выравнивание текста по правому краю
    \bfseries УТВЕРЖДЕНО \\
    Заместитель директора \\
    \rule{2cm}{0.4pt}/С.Н. Нагиева \\
    от «25» августа 2025 г.
    \end{tabular}
\end{flushright}

% 3. ФИКСИРОВАННЫЙ отступ до заголовка (поднимаем его выше)
\vspace*{2.5cm} % ← ключевая настройка: звёздочка гарантирует отступ в начале страницы

% 4. Название документа (по центру)
\begin{center}
  {\bfseries КОНТРОЛЬНО-ОЦЕНОЧНЫЕ СРЕДСТВА}\\
  {\bfseries ПРОМЕЖУТОЧНОЙ АТТЕСТАЦИИ}\\
  {\bfseries \readline{1}{parameters.txt} }\\
  {\bfseries \readline{3}{parameters.txt} \readline{4}{parameters.txt}}
\end{center}

% 5. Отступ до года (оставляем гибким, чтобы год был внизу)
\vfill

% 6. Год (по центру, внизу)
\begin{center}
  {2025}
\end{center}
\clearpage

% Основной текст документа

{\bfseries Рассмотрено и одобрено на заседании}

Предметной цикловой комиссией

{\itshape Информационные технологии}

Протокол №8

от 17 марта 2021г.

\vspace{\baselineskip}

Председатель ПЦК

\rule{2cm}{0.4pt} Н.В. Кадочникова

\vspace{3\baselineskip}

{\bfseries Разработчик:}

ГБПОУ «Пермский политехнический колледж имени Н.Г. Славянова»

{\bfseries Быстров Никита Олегович, преподаватель}

\clearpage

\section*{Пояснительная записка}

КОС промежуточной аттестации предназначены для контроля и оценки образовательных достижений студентов, проходящих \readline{2}{parameters.txt}\unskip{}.

КОС разработаны в соответствии требованиями ООП СПО по профессии \readline{3}{parameters.txt}\unskip{}, квалификации Сетевой и системный администратор, рабочей программы МДК.

Учебная практика осваивается в течение 6 семестра в объеме 162 часов.
КОС включает контрольные материалы для проведения промежуточной аттестации в форме: экзамена.

По результатам изучения \readline{2}{parameters.txt}\unskip{} студент должен:

уметь:
\begin{itemize}
  \foreach \n in {1,...,3} {
    \item[--] \readline{\n}{ymet.txt}
  }
\end{itemize}

знать:
\begin{itemize}
  \foreach \n in {1,...,2} {
    \item[--] \readline{\n}{znat.txt}
  }
\end{itemize}

КОС промежуточной аттестации имеют своей целью определение сформированности общих и профессиональных компетенций:

\printcompetencylist{competencies.csv}

\clearpage

\section*{Контрольно-оценочные средства промежуточной аттестации}

{\bfseries Форма промежуточной аттестации}: Экзамен (по билетам):

\begin{center}
  {\bfseries Вопросы для подготовки к экзамену}
\end{center}


\begin{enumerate}
    \item Планирование и реализация хранилищ данных.
    \item Планирование и реализация защиты сетей.
    \item Проектирование и реализация защиты служб доступа к сети.
    \item Проектирование и внедрение стратегии групповых политик.
    \item Планирование и развертывание серверов.
    \item Планирование и реализация стратегии виртуализации серверов.
    \item Планирование и реализация сетевой инфраструктуры и систем хранения данных для виртуализации.
    \item Планирование и реализация решения по администрированию виртуализации.
    \item Планирование и реализация стратегии мониторинга серверов.
    \item Планирование и реализация решений высокой доступности для файловых служб и приложений.
    \item Планирование и реализация инфраструктуры открытых ключей.
    \item Планирование и реализация доступа к данным для пользователей и устройств.
\end{enumerate}

\clearpage

% Автобилеты из папки
\foreach \n in {1,...,32} {
  \IfFileExists{examtickets/t\n.txt}{
    \examticket{\n}{\input{examtickets/t\n.txt}}
  }{}
}

\clearpage

Критерии оценки:

\vspace{\baselineskip}

\input{evaluation.txt}

\end{document}
