\documentclass[10pt]{article}
\usepackage[T2A]{fontenc}
\usepackage[utf8]{inputenc}
\usepackage[russian]{babel}
\usepackage{setspace}
\usepackage{titlesec}
\usepackage{enumitem}
\usepackage{pgffor}     % для цикла \foreach
\usepackage{geometry}
\usepackage{indentfirst} % делает отступ первой строки после \section
\usepackage{needspace}  % для проверки свободного места
\usepackage{longtable}
\usepackage{array}
\usepackage{pdflscape}
\usepackage{multirow}
\usepackage{url} % Для корректного отображения URL

\usepackage{titlesec}

% Выравнивание ВСЕХ заголовков по левому краю
\titleformat{\part}{\normalfont\huge\bfseries}{\thepart}{1em}{}
\titleformat{\chapter}{\normalfont\Large\bfseries}{\thechapter}{1em}{}
% \titleformat{\section}[hang]{\normalfont\large\bfseries}{}{1em}{}
\titleformat{\section}{\normalfont\large\bfseries\centering}{}{1em}{}
\titleformat{\subsection}{\normalfont\bfseries}{\thesubsection}{1em}{}
\titleformat{\subsubsection}{\normalfont\bfseries}{\thesubsubsection}{1em}{}
\titleformat{\paragraph}[runin]{\normalfont\bfseries}{\theparagraph}{1em}{}
\titleformat{\subparagraph}[runin]{\normalfont\bfseries}{\thesubparagraph}{1em}{}

% Убираем отступы слева у всех заголовков
\titlespacing*{\part}{0pt}{50pt}{40pt}
\titlespacing*{\chapter}{0pt}{50pt}{20pt}
\titlespacing*{\section}{0pt}{3.5ex plus 1ex minus .2ex}{2.3ex plus .2ex}
\titlespacing*{\subsection}{0pt}{3.25ex plus 1ex minus .2ex}{1.5ex plus .2ex}
\titlespacing*{\subsubsection}{0pt}{3.25ex plus 1ex minus .2ex}{1.5ex plus .2ex}
\titlespacing*{\paragraph}{0pt}{3.25ex plus 1ex minus .2ex}{1em}
\titlespacing*{\subparagraph}{0pt}{3.25ex plus 1ex minus .2ex}{1em}

\onehalfspacing
\setlist[itemize]{noitemsep, topsep=4pt, parsep=0pt, partopsep=0pt, leftmargin=1.25cm}
\setlist[enumerate]{noitemsep, topsep=4pt, parsep=0pt, partopsep=0pt, leftmargin=1.25cm}
\setitemize[1]{label=--}
\pagestyle{empty}
\setlength{\parindent}{1.25cm} % стандартный отступ по ГОСТ (1.25 см)
\setlength{\parskip}{0pt}      % убрать вертикальный отступ между абзацами
\geometry{top=2cm, bottom=2cm, left=2.5cm, right=2cm} % Поля страницы

%парсер тхт файлов
\newread\myfile
\def\readline#1#2{%
  \openin\myfile=#2\relax
  \readlinehelper{#1}{1}%
  \closein\myfile
}
\def\readlinehelper#1#2{%
  \ifeof\myfile
    % Если дошли до конца — ничего не делаем
  \else
    \read\myfile to \tmp
    \ifnum#2=#1\relax
      \tmp % выводим нужную строку
    \else
      \expandafter\readlinehelper\expandafter#1\expandafter{\number\numexpr#2+1\relax}%
    \fi
  \fi
}

\makeatletter

% Накопитель
\def\competencylist{}

% Чтение CSV
\newread\compfile
\def\readcompetencies#1{%
  \openin\compfile=#1
  % Пропускаем первую строку (заголовок)
  \read\compfile to \dummyline
  % Читаем остальные
  \loop
    \read\compfile to \line
    \unless\ifeof\compfile
      \ifx\line\empty\else
        \expandafter\extractfirstfield\line\relax
      \fi
  \repeat
  \closein\compfile
}

% Извлекает всё до первой запятой
\def\extractfirstfield#1,#2\relax{%
  \ifx\competencylist\@empty
    \gdef\competencylist{#1}%
  \else
    \g@addto@macro\competencylist{, #1}%
  \fi
}

\makeatother

\makeatletter

% Вторая функция: вывод списка "comp. description"
\def\competencyitems{}

\def\readcompetencyitems#1{%
  \openin\compfile=#1
  % Пропускаем заголовок
  \read\compfile to \dummyline
  \loop
    \read\compfile to \line
    \unless\ifeof\compfile
      \ifx\line\empty\else
        \expandafter\parseitem\line\relax
      \fi
  \repeat
  \closein\compfile
}

% Парсинг строки: "OK 01,Описание" → "OK 01. Описание"
\def\parseitem#1,#2\relax{%
  \ifx\competencyitems\@empty
    \gdef\competencyitems{#1. #2}%
  \else
    \g@addto@macro\competencyitems{\par #1. #2}%
  \fi
}

\makeatother

% Удобный макрос для использования
\newcommand{\printcompetencylist}[1]{%
  \readcompetencyitems{#1}%
  \competencyitems
}

\begin{document}

% 1. Шапка — название колледжа (по центру)
\begin{center}
  {
  государственное бюджетное профессиональное образовательное учреждение\\
  «Пермский политехнический колледж имени Н.Г. Славянова»\\
  Предметная цикловая комиссия Информационные технологии
  }
\end{center}

% 2. Блок УТВЕРЖДЕНО (слева страницы, текст внутри — по правому краю)
\begin{flushright}
    \vspace{1.5cm}
    \begin{tabular}{r@{}} % r = выравнивание текста по правому краю
    \bfseries УТВЕРЖДЕНО \\
    Директор \\
    \rule{2cm}{0.4pt}/А.Н. Попов \\
    от «25» августа 2025 г.
    \end{tabular}
\end{flushright}

% 3. ФИКСИРОВАННЫЙ отступ до заголовка
\vspace*{2.5cm} % ← ключевая настройка: звёздочка гарантирует отступ в начале страницы

% 4. Название документа (по центру)
\begin{center}
  {\bfseries РАБОЧАЯ ПРОГРАММА УЧЕБНОЙ ДИСЦИПЛИНЫ}\\
  {\bfseries \readline{1}{parameters.txt}\unskip{}}\\
  {для реализации Программы подготовки специалистов среднего звена (ППССЗ)}\\
  {по специальности {\itshape \readline{3}{parameters.txt}\unskip{} \readline{4}{parameters.txt}\unskip{}} (в рамках ТОП-50)}
\end{center}

% 5. Отступ до года (оставляем гибким, чтобы год был внизу)
\vfill

% 6. Год (по центру, внизу)
\begin{center}
  {2025}
\end{center}
\clearpage

% Основной текст документа ===================================================================================================

Рабочая программа учебной дисциплины \readline{2}{parameters.txt}\unskip{} разработана на основе:

- Федерального государственного образовательного стандарта среднего профессионального образования (ФГОС СПО) по специальности \readline{3}{parameters.txt}\unskip{} \readline{4}{parameters.txt}\unskip{} (приказ Минобрнауки России от 09декабря 2016г. №1548 (Зарег. Минюсте России 26 декабря 2016 г. N 44978);

- Методических рекомендаций по реализации ФГОС СПО по 50 наиболее востребованным и перспективным профессиям и специальностям (Письмо Департамента государственной политики в сфере подготовки рабочих кадров и ДПО Минобнауки России от 29.02.2017г. №06-156); 

- Учебного плана ГБПОУ «ППК им. Н.Г. Славянова», утвержденного директором колледжа 23 марта 2017г.

\vspace{\baselineskip}

{\bfseries Рассмотрено и одобрено на заседании}

Предметной цикловой комиссией

{\itshape Информационные технологии}

Протокол №8

от 17 марта 2021г.

\vspace{\baselineskip}

Председатель ПЦК

\rule{2cm}{0.4pt} Н.В. Кадочникова

\vspace{2\baselineskip}

{\bfseries Разработчик:}

ГБПОУ «Пермский политехнический колледж имени Н.Г. Славянова»

{\bfseries Быстров Никита Олегович}, преподаватель

\vspace{3\baselineskip}

{\bfseries Рекомендована к утверждению}

Методическим советом ГБПОУ «Пермского политехнического колледжа имени Н.Г. Славянова»

Заключение Методического совета Протокол №8 от 22 марта 2017г.

\vspace{\baselineskip}

Зам. директора по УМР \rule{2cm}{0.4pt}С.Н. Нагиева

\clearpage

\section*{Содержание}

\begin{tabular}{|l|r|}
\hline
 & \textbf{Стр.} \\
\hline
1. Общая характеристика рабочей программы учебной дисциплины & \pageref{sec:obshchaya} \\
\hline
2. Структура и содержания учебной дисциплины & \pageref{sec:struktura} \\
\hline
3. Условия реализации программы & \pageref{sec:usloviya} \\
\hline
4. Контроль и оценка результатов освоения учебной дисциплины & \pageref{sec:kontrol} \\
\hline
5. Возможности использования программы в других ППССЗ & \pageref{sec:vozmozhnosti} \\
\hline
\end{tabular}

\clearpage

\section{ОБЩАЯ ХАРАКТЕРИСТИКА РАБОЧЕЙ ПРОГРАММЫ УЧЕБНОЙ ДИСЦИПЛИНЫ \\
\readline{2}{parameters.txt}\unskip{}}
\label{sec:obshchaya}

\subsection{Область применения рабочей программы}

Рабочая программа учебной дисциплины является частью программы подготовки специалистов среднего звена (ППССЗ) по специальности СПО в соответствии с ФГОС СПО по ТОП-50 \readline{3}{parameters.txt}\unskip{} \readline{4}{parameters.txt}\unskip{}, утверждённым приказом Министерства образования и науки Российской Федерации 09 декабря 2016 № 1548, зарегистрированным в Министерстве юстиции Российской Федерации 26 декабря 2016 года, регистрационный № 44978, укрупнённой группыспециальностей 09.00.00 Информатика и вычислительная техника.

\subsection{Место дисциплины в структуре основной профессиональной образовательной программы:}

Учебная дисциплина \readline{2}{parameters.txt}\unskip{} относится к общепрофессиональному учебному циклу (ОП.00) ППССЗ специальности \readline{3}{parameters.txt}\unskip{}.

\subsection{Цели и планируемые результаты освоения дисциплины:}

В результате освоения дисциплины обучающийся должен

уметь:
\begin{itemize}
  \foreach \n in {1,...,3} {
    \item[--] \readline{\n}{ymet.txt}
  }
\end{itemize}

знать:
\begin{itemize}
  \foreach \n in {1,...,2} {
    \item[--] \readline{\n}{znat.txt}
  }
\end{itemize}

В результате изучения дисциплины обучающийся осваивает элементы общих и профессиональных компетенций.

\printcompetencylist{competencies.csv}

\clearpage

\section{СТРУКТУРА ПРИМЕРНОЙ УЧЕБНОЙ ДИСЦИПЛИНЫ  }
\label{sec:struktura}

\subsection{Объем учебной дисциплины и виды учебной работы}

\begin{tabular}{|p{10cm}|c|}
\hline
\textbf{Вид учебной работы} & Объем часов \\
\hline
\textbf{Объем образовательной программы} & 96 \\
\hline
\multicolumn{2}{|l|}{в том числе:} \\
\hline
самостоятельная работа обучающихся & 16 \\
\hline
консультации & 8 \\
\hline
теоретическое обучение & 20 \\
\hline
практические занятия & 44 \\
\hline
лабораторные занятия & - \\
\hline
курсовая работа (проект) & - \\
\hline
контрольная работа & 6 \\
\hline
промежуточная аттестация: дифференцированный зачет & 2 \\
\hline
\textbf{Объем практической подготовки} & 44 \\
\hline
\end{tabular}

\clearpage

\begin{landscape}

\subsection{Тематический план и содержание учебной дисциплины ОП.04 Основы алгоритмизации и программирования}

    \small
    \setlength{\tabcolsep}{4pt}
    \renewcommand{\arraystretch}{1.3}
    
    \begin{longtable}{|
      >{\raggedright\arraybackslash}p{3.0cm}|
      >{\raggedright\arraybackslash}p{12.0cm}|
      >{\centering\arraybackslash}p{2.0cm}|
      >{\centering\arraybackslash}p{2.0cm}|
      >{\raggedright\arraybackslash}p{3.0cm}|
    }
    
    % Заголовок первой страницы
    \hline
    \multicolumn{1}{|>{\centering\arraybackslash}p{3.0cm}|}{\textbf{Наименование разделов и тем}} &
    \multicolumn{1}{>{\centering\arraybackslash}p{12.0cm}|}{\textbf{Содержание учебного материала и формы организации деятельности обучающихся}} &
    \multicolumn{1}{>{\centering\arraybackslash}p{2.0cm}|}{\textbf{Уровень освоения}} &
    \multicolumn{1}{>{\centering\arraybackslash}p{2.0cm}|}{\textbf{Объем часов}} &
    \multicolumn{1}{>{\centering\arraybackslash}p{3.0cm}|}{\textbf{Осваиваемые элементы компетенций}} \\
    \hline
    \multicolumn{1}{|>{\centering\arraybackslash}p{3.0cm}|}{1} &
    \multicolumn{1}{>{\centering\arraybackslash}p{12.0cm}|}{2} &
    \multicolumn{1}{>{\centering\arraybackslash}p{2.0cm}|}{3} &
    \multicolumn{1}{>{\centering\arraybackslash}p{2.0cm}|}{4} &
    \multicolumn{1}{>{\centering\arraybackslash}p{3.0cm}|}{5} \\
    \hline
    \endfirsthead
    
    % Заголовок последующих страниц
    \hline
    \multicolumn{1}{|>{\centering\arraybackslash}p{3.0cm}|}{\textbf{Наименование разделов и тем}} &
    \multicolumn{1}{>{\centering\arraybackslash}p{12.0cm}|}{\textbf{Содержание учебного материала и формы организации деятельности обучающихся}} &
    \multicolumn{1}{>{\centering\arraybackslash}p{2.0cm}|}{\textbf{Уровень освоения}} &
    \multicolumn{1}{>{\centering\arraybackslash}p{2.0cm}|}{\textbf{Объем часов}} &
    \multicolumn{1}{>{\centering\arraybackslash}p{3.0cm}|}{\textbf{Осваиваемые элементы компетенций}} \\
    \hline
    \multicolumn{1}{|>{\centering\arraybackslash}p{3.0cm}|}{1} &
    \multicolumn{1}{>{\centering\arraybackslash}p{12.0cm}|}{2} &
    \multicolumn{1}{>{\centering\arraybackslash}p{2.0cm}|}{3} &
    \multicolumn{1}{>{\centering\arraybackslash}p{2.0cm}|}{4} &
    \multicolumn{1}{>{\centering\arraybackslash}p{3.0cm}|}{5} \\
    \hline
    \endhead
    
    % Подвал последней страницы
    \hline
    \endlastfoot
    
    % === ТЕЛО ТАБЛИЦЫ ===
    
    \multicolumn{5}{|l|}{\textbf{Раздел 1. Основы баз данных}} \\
    \hline
    
    \multirow{2}{3.0cm}{\textbf{Тема 1.1 Основные понятия баз данных}} &
    Основные понятия теории БД &
    1 & 1 &
    \multirow{12}{3.0cm}{ОК 01, ОК 02, ОК 03, ОК 04, ОК 05, ОК 09, ОК 10, ПК 1.2, ПК 1.5} \\
    \cline{2-4}
     &
    Логическая и физическая независимость данных. Типы моделей данных. Нормализация реляционной БД. Освоение принципов проектирования БД. Преобразование реляционной БД в сущности и связи. Проектирование реляционной БД. Нормализация таблиц. &
    1 & 1 &
     \\
    \cline{1-4}
    
    \multirow{2}{3.0cm}{\textbf{Тема 1.2 Взаимосвязи в моделях}} &
    Реляционная модель &
    1 & 2 &
     \\
    \cline{2-4}
     &
    Проектирование баз данных &
    1 & 2 &
     \\
    \cline{2-4}
    
    \multirow{9}{3.0cm}{\textbf{Тема 1.3 Проектирование структур баз данных}} &
    Проектирование базы данных &
    2 & 6 &
     \\
    \cline{1-4}
     &
    Установка соединения с сервером Microsoft SQL Server и принципы создания баз данных &
    2 & 6 &
     \\
    \cline{2-4}
     &
    Проектирование таблиц и определение ограничений &
    2 & 6 &
     \\
    \cline{2-4}
     &
    Введение в язык SQL. Создание таблиц и ограничений на SQL &
    2 & 6 &
     \\
    \cline{2-4}
     &
    Создание запросов на выборку. Отбор строк по условию &
    2 & 4 &
     \\
    \cline{2-4}
     &
    Создание многотабличных запросов. Запросы на соединение &
    2 & 4 &
     \\
    \cline{2-4}
     &
    Создание запросов на группировку и сортировку данных. Запросы на изменение. Использование встроенных функций. &
    2 & 4 &
     \\
    \cline{2-4}
     &
    Выполнение практических работ &
    2 & 18 &
     \\
    \hline
    
    \multicolumn{3}{|r|}{Консультации} & 4 & \\
    \hline
    
    \multicolumn{3}{|r|}{Экзамен} & 8 & \\
    \hline
    
    \multicolumn{3}{|r|}{\textbf{Всего:}} & \textbf{72} & \\
    
    \end{longtable}

Для характеристики уровня освоения учебного материала используются следующие обозначения:

1 – ознакомительный (воспроизведение информации, узнавание (распознавание), объяснение ранее изученных объектов, свойств и т.п.); 

2 – репродуктивный (выполнение деятельности по образцу, инструкции или под руководством); 

3 – продуктивный (самостоятельное планирование и выполнение деятельности, решение проблемных задач).

\end{landscape}

\clearpage

\section{УСЛОВИЯ РЕАЛИЗАЦИИ ПРОГРАММЫ }
\label{sec:usloviya}

\subsection{Материально-техническое обеспечение}
Реализация программы предполагает наличие учебной лаборатории Информационных технологий, Программирования и баз данных
Оборудование лаборатории и рабочих мест лаборатории: 

\begin{itemize}
  \item[*] сетевой компьютерный класс с выходом в Интернет,
  \item[*] комплект посадочных мест по количеству  обучающихся;
  \item[*] рабочее место преподавателя с компьютером;
  \item[*] комплект учебно-методических пособий по дисциплине;
  \item[*] программное обеспечение по дисциплине.
\end{itemize}

Технические средства обучения:

\begin{itemize}
  \item[*] интерактивная доска;
  \item[*] проектор;
  \item[*] принтер лазерный (принтер лазерный сетевой);
  \item[*] источник бесперебойного питания;
  \item[*] сканер, цифровой фотоаппарат, Web-камера;
  \item[*] аудиторная доска для письма фломастером с магнитной поверхностью;
  \item[*] шкафы для хранения оборудования;
  \item[*] демонстрационные печатные пособия и демонстрационные ресурсы в электронном представлении.
\end{itemize}

\subsection{Информационное обеспечение обучения}

\textbf{Основные источники:}

\begin{enumerate}[leftmargin=*,label=\arabic*.,nosep]
  \item Голицына О.Л., Попов И.И. Основы алгоритмизации и программирования: учебное пособие. — 3-е изд. — М.: Форум, 2020. — Гриф Минобрнауки РФ.

  \item Семакин И.Г., Шестаков А.П. Основы алгоритмизации и программирования: учебник для студентов учреждений среднего профессионального образования. — М.: Издательский центр «Академия», 2018. — 324 с. — Рекомендовано ФИРО.

  \item Семакин И.Г., Шестаков А.П. Основы алгоритмизации и программирования. Практикум: учебное пособие для студентов учреждений среднего профессионального образования. — М.: Издательский центр «Академия», 2018. — 154 с. — Рекомендовано ФИРО.
\end{enumerate}

\vspace{0.5em}
\textbf{Дополнительные источники:}

\begin{enumerate}[leftmargin=*,label=\arabic*.,nosep]
  \item Кузьменко В.Г. Базы данных в Visual Basic и VBA. Самоучитель. — М.: ООО «Бином-Пресс», 2019.
\end{enumerate}

\vspace{0.5em}
\textbf{Интернет-ресурсы:}

\begin{enumerate}[leftmargin=*,label=\arabic*.,nosep]
  \item Канцедал С.А. Алгоритмизация и программирование: учебное пособие. — М.: ИД ФОРУМ: НИЦ ИНФРА-М, 2019. — 352 с. — (Профессиональное образование). — ISBN 978-5-8199-0355-1. — \url{http://znanium.com/catalog.php?bookinfo=429576}

  \item Гагарина Л.Г. (ред.). Основы алгоритмизации и программирования: учебное пособие. — М.: ИД ФОРУМ: ИНФРА-М, 2017. — 416 с. — (Профессиональное образование). — ISBN 978-5-8199-0279-0. — \url{http://znanium.com/catalog.php?item=bookinfo&book=336649}

  \item Колдаев В.Д. Структуры и алгоритмы обработки данных: учебное пособие. — М.: ИЦ РИОР: НИЦ ИНФРА-М, 2019. — 296 с. — (Высшее образование: Бакалавриат). — ISBN 978-5-369-01264-2. — \url{http://znanium.com/catalog.php?item=bookinfo&book=418290}
\end{enumerate}

\clearpage

\section*{КОНТРОЛЬ И ОЦЕНКА РЕЗУЛЬТАТОВ ОСВОЕНИЯ УЧЕБНОЙ ДИСЦИПЛИНЫ  }
\label{sec:kontrol}

\begin{tabular}{|p{6.0cm}|p{5.5cm}|p{3.0cm}|}
\hline
\textbf{Результаты обучения} & \textbf{Критерии оценки} & \textbf{Формы и методы оценки} \\
\hline
\textbf{Перечень знаний, осваиваемых в рамках дисциплины} & & \\
\hline
Понятие алгоритмизации, свойства алгоритмов, общие принципы построения алгоритмов, основные алгоритмические конструкции & \multirow{5}{5.5cm}{90--100\,\% правильных ответов – «5»;\\ 70--89\,\% правильных ответов – «4»;\\ 50--69\,\% правильных ответов – «3»;\\ менее 50\,\% правильных ответов – «2»} & \multirow{5}{3.0cm}{устный опрос, тестирование, выполнение индивидуальных заданий различной сложности Промежуточная аттестация: экзамен} \\
\cline{1-1}
Эволюция языков программирования, их классификация, понятие системы программирования & & \\
\cline{1-1}
Основные элементы языка, структура программы, операторы и операции, управляющие структуры, структуры данных, файлы, классы памяти & & \\
\cline{1-1}
Подпрограммы, составление библиотек подпрограмм & & \\
\cline{1-1}
Объектно-ориентированная модель программирования, основные принципы ООП: понятие классов и объектов, их свойств и методов, инкапсуляции, полиморфизма, наследования и переопределения & & \\
\hline
\textbf{Перечень умений, осваиваемых в рамках дисциплины} & & \\
\hline
Разрабатывать алгоритмы для конкретных задач & \multirow{7}{5.5cm}{Правильность и полнота выполнения заданий, точность формулировок и расчётов, соответствие требованиям;\\ адекватность и оптимальность выбора методов и последовательностей действий;\\ точность оценки;\\ соответствие требованиям инструкций и регламентов;\\ рациональность действий} & \multirow{7}{3.0cm}{защита отчетов по практическим занятиям;
экспертная оценка демонстрируемых умений по составлению программ} \\
\cline{1-1}
Использовать программы для графического отображения алгоритмов & & \\
\cline{1-1}
Определять сложность работы алгоритмов & & \\
\cline{1-1}
Работать в среде программирования & & \\
\cline{1-1}
Реализовывать построенные алгоритмы в виде программ на конкретном языке программирования & & \\
\cline{1-1}
Оформлять код программы в соответствии со стандартом кодирования & & \\
\cline{1-1}
Выполнять проверку и отладку кода программы & & \\
\hline
\end{tabular}

\section*{ВОЗМОЖНОСТИ ИСПОЛЬЗОВАНИЯ ПРОГРАММЫ В ДРУГИХ ППССЗ}
\label{sec:vozmozhnosti}

Учебная дисциплина ОП.04 Основы алгоритмизации и программирования может быть использована для обучения укрупненной группы профессий и специальностей 09.00.00 Информатика и вычислительная техника

\end{document}
